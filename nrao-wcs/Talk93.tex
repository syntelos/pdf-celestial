\documentstyle [twoside]{article}
\title{Celestial and Other Coordinates in FITS}
\author{1-June-93}
\date{\ }
\parskip 4mm
\special{landscape()}
\linewidth 10in
\textwidth 10in                     % text width excluding margin
\textheight 7.3 in
\marginparsep 0in
\oddsidemargin -.45in                 % EWG from -.25
\evensidemargin -.45in
\topmargin -.5in
\headsep 0.25in
\headheight 0.25in
\parindent 0in
%
%
\begin{document}

\newcommand{\normalstyle}{\baselineskip 4mm \parskip 2mm \normalsize}
\newcommand{\tablestyle}{\baselineskip 2mm \parskip 1mm \small }
\newcommand{\eg}{{\it e.g.,}}
\newcommand{\ie}{{\it i.e.,}}
\pagestyle{myheadings}
\thispagestyle{empty}

%
%

\vskip -.5cm
\pretolerance 10000 
\listparindent 0cm
\labelsep 0cm
%
%

\vskip -30pt
%\maketitle
\vskip -30pt
\normalstyle
\pagestyle{empty}
\huge

\hphantom{aaa}
\vskip 25pt

\centerline{\Huge Representations of Celestial Coordinates in FITS}
\vskip 50pt
\centerline{\Huge Eric W. Greisen}
\vskip 20pt
\centerline{\huge National Radio Astronomy Observatory}
\vskip 30pt
\centerline{\Huge Mark Calabretta}
\vskip 20pt
\centerline{\huge Australia Telescope National Facility}
\vfill\eject

\hphantom{aaa}
\vskip -25pt

\centerline{\Huge Initial FITS Coordinate Specification}
\vskip 20pt
\begin{center}
\begin{tabular}{l}
\multicolumn{1}{l}{\Huge Keywords} \\
\noalign{\vskip 6pt}
\hphantom{aa} {\tt CTYPE}{\it n}  Coordinate type (8 characters)\\
\hphantom{aa} {\tt CRPIX}{\it n}  Reference pixel location \\
\hphantom{aa} {\tt CRVAL}{\it n}  Coordinate value at reference pixel \\
\hphantom{aa} {\tt CDELT}{\it n}  Coordinate increment at reference pixel \\
\hphantom{aa} {\tt CROTA}{\it n}  Coordinate ``rotation'' \\
\noalign{\vskip 13pt}
\multicolumn{1}{l}{\Huge Problems with this} \\
\noalign{\vskip 6pt}
\hphantom{aa} $\bullet$  Types inadequate for celestial coordinates,
                             velocities\\ 
\hphantom{aa} $\bullet$  Physical meaning of reference pixel undefined \\
\hphantom{aa} $\bullet$  Rotation undefined, inadequate for general
                             use \\
\hphantom{aa} $\bullet$  No provision for skew in images \\
\noalign{\vskip 13pt}
\multicolumn{1}{l}{\Huge AIPS enhancements} \\
\noalign{\vskip 6pt}
\hphantom{aa} $\bullet$  Types defined for some celestial coordinates,
                            velocities \\
\hphantom{aa} $\bullet$  Reference pixel defined as tangent point\\
\hphantom{aa} $\bullet$  Rotation limited to celestial coordinates \\
\hphantom{aa} $\bullet$  Still no skew or offset rotations  \\
\hphantom{aa} $\bullet$  Limited view of projective geometries
\end{tabular}
\end{center}
\vfill\eject

\hphantom{aaa}
\vskip -25pt

\centerline{\Huge Proposed FITS Coordinate Specification}
\vskip 20pt
\centerline{\Huge Keywords for All Coordinate Types}
\vskip 20pt
\begin{center}
\begin{tabular}{l}
\multicolumn{1}{l}{ {\tt CD{\it n}\_{\it m}} $ = $ {\tt CD}{\it
      nnn}{\it mmm} {\Huge Matrix}} \\
\hphantom{aa} $\bullet$  Replaces {\tt CDELT}{\it n} and {\tt
                            CROTA}{\it n} keywords \\
\hphantom{aa} $\bullet$  Allows skew, offset and general rotations \\
\hphantom{aa} $\bullet$  Allows dissimilar coordinates to be rotated
                            together \\
\hphantom{aa} $\bullet$  Relative coordinates given by linear
                            transform :
\end{tabular}
\end{center}
\begin{displaymath}
\left ( \begin{array}{c} x \\ y \\ z \\ \vdots \end{array} \right ) =
   \left ( \begin{array}{cccc}
   \hbox{{\tt CD1\_1}} & \hbox{{\tt CD1\_2}} & \hbox{{\tt CD1\_3}} &
         \ldots \\ 
   \hbox{{\tt CD2\_1}} & \hbox{{\tt CD2\_2}} & \hbox{{\tt CD2\_3}} &
         \ldots \\ 
   \hbox{{\tt CD3\_1}} & \hbox{{\tt CD3\_2}} & \hbox{{\tt CD3\_3}} &
         \ldots \\ 
   \vdots          & \vdots          & \vdots          & \ddots
   \end{array} \right )
   \left ( \begin{array}{c}
   i - i_0 \\ j - j_0 \\ k - k_0 \\ \vdots \end{array} \right )
\end{displaymath}
\vfill\eject

\hphantom{aaa}
\vskip -25pt

\centerline{\Huge Proposed FITS Coordinate Specification}
\vskip 20pt
\centerline{\Huge Keywords for Celestial Coordinates}
\vskip 20pt
\begin{center}
\begin{tabular}{l}
\multicolumn{1}{l}{{\tt CTYPE{\it n}} {\Huge format defined}} \\
\hphantom{aa} $\bullet$  First 4 characters give type of ``standard
                           coordinate system'' as \\
\hphantom{aa $\bullet$ aa} equatorial --- {\tt RA--} and {\tt DEC-} \\
\hphantom{aa $\bullet$ aa} galactic --- {\tt GLON} and {\tt GLAT} \\
\hphantom{aa $\bullet$ aa} ecliptic --- {\tt ELON} and {\tt ELAT} \\
\hphantom{aa} $\bullet$  Second 4 characters give type of projection
                            as {\tt -{\it ccc}} \\
\hphantom{aa $\bullet$ aa} where {\it ccc} defined by convention, such
                           as \\
\hphantom{aa $\bullet$ aa} {\tt SIN}, {\tt TAN}, {\tt ARC}, {\tt AIT} \\
\noalign{\vskip 12pt}
\multicolumn{1}{l}{{\tt CRPIX{\it n}} {\Huge meaning defined by
                           projection }} \\
\hphantom{aa} $\bullet$  Each projection has ``native coordinate
                           system '' \\
\hphantom{aa} $\bullet$  Reference pixel is native north pole
                           $(0,90^{\circ})$ for \\
\hphantom{aa $\bullet$ aa} azimuthal and conical projections \\
\hphantom{aa} $\bullet$  Reference pixel is native origin $(0,0)$ for \\
\hphantom{aa $\bullet$ aa} cylindrical and conventional projections \\
\end{tabular}
\end{center}
\vfill\eject

\hphantom{aaa}
\vskip -25pt

\centerline{\Huge Proposed FITS Coordinate Specification}
\vskip 20pt
\centerline{\Huge Celestial Coordinates Continued}
\vskip 20pt
\begin{center}
\begin{tabular}{l}
\multicolumn{1}{l}{ {\tt CRVAL{\it n}} {\Huge clarified}} \\
\hphantom{aa} $\bullet$  Value in standard coordinates at the
                            reference pixel \\ 
\hphantom{aa} $\bullet$  {\tt LONGPOLE} keyword added for generality \\
\hphantom{aa $\bullet$ aa} Native longitude of north pole of standard
                              system \\
\hphantom{aa $\bullet$ aa} Default value is $180^{\circ}$ \\
\noalign{\vskip 10pt}
\multicolumn{1}{l}{{\tt PROJP{\it j}} {\Huge keywords added to define
                           some projections }} \\
\noalign{\vskip 10pt}
\multicolumn{1}{l}{\Huge Astrometry-related keywords added} \\
\hphantom{aa} $\bullet$ {\tt EQUINOX} replaces {\tt EPOCH} for the
                 epoch of the mean \\
\hphantom{aa $\bullet$ aa}  equator and equinox in years \\
\hphantom{aa} $\bullet$ {\tt MJD-OBS} modified Julian date of
                 observation in days \\
\hphantom{aa} $\bullet$ {\tt RADECSYS} frame of reference of
                 equatorial coordinates \\
\hphantom{aa $\bullet$ aa} as {\tt FK4}, {\tt FK4-NO-E}, {\tt FK5},
                 {\tt GAPPT} \\
\end{tabular}
\end{center}
\vfill\eject

\hphantom{aaa}
\vskip -25pt

\centerline{\Huge Proposed FITS Coordinate Specification}
\vskip 20pt
\centerline{\Huge Conventions and Matters of ``Good Form''}
\vskip 20pt
\begin{center}
\begin{tabular}{rl}
1. & The center of each pixel is its location. \\
\noalign{\vskip 5pt}
2. & Default viewing convention: \\
   &\hphantom{a} $\bullet$ First pixel at lower left corner, \\
   &\hphantom{a} $\bullet$ First axis displayed along horizontal, \\
   &\hphantom{a} $\bullet$ Second axis displayed along vertical. \\
\noalign{\vskip 5pt}
3. & Diagonal elements of {\tt CD{\it i}\_{\it j}} should predominate.\\ 
   &\hphantom{a} $\bullet$ Do not hide transpositions in the {\tt CD}
                             matrix. \\
\noalign{\vskip 5pt}
4. & Forbid rotation into axes which have only integral values.\\
\noalign{\vskip 5pt}
5. & {\tt NCP} projection (of WSRT) changed to offset {\tt SIN}
        projection.\\ 
\noalign{\vskip 5pt}
6. & When possible, {\tt CDELT}{\it n} should be written along with
        {\tt CD{\it i}\_{\it j}}. \\
\noalign{\vskip 5pt}
7. & For longitude axis $i$ and latitude axis $j$, the conversion is\\
\end{tabular}
\end{center}
\vskip -5pt
\begin{displaymath}
\left ( \begin{array}{cc}
   \hbox{{\tt CD{\it i}\_{\it i}}} & \hbox{{\tt CD{\it i}\_{\it j}}} \\
   \hbox{{\tt CD{\it j}\_{\it i}}} & \hbox{{\tt CD{\it j}\_{\it j}}}
   \end{array} \right ) =
\left ( \begin{array}{cc}
   \hbox{{\tt CDELT{\it i}}} \cos(\hbox{{\tt CROTA{\it j}}}) &
   - \hbox{{\tt CDELT{\it j}}} \sin(\hbox{{\tt CROTA{\it j}}}) \\
   \hbox{{\tt CDELT{\it i}}} \sin(\hbox{{\tt CROTA{\it j}}}) &
   \hbox{{\tt CDELT{\it j}}} \cos(\hbox{{\tt CROTA{\it j}}})
   \end{array} \right ) \, .
\end{displaymath}
\vfill\eject

\end{document}
