\documentstyle [twoside]{article}
\title{Celestial and Other Coordinates in FITS}
\author{1-June-93}
\date{\ }
\parskip 4mm
\special{landscape()}
\linewidth 10in
\textwidth 10in                     % text width excluding margin
\textheight 7.3 in
\marginparsep 0in
\oddsidemargin -.45in                 % EWG from -.25
\evensidemargin -.45in
\topmargin -.5in
\headsep 0.25in
\headheight 0.25in
\parindent 0in
%
%
\begin{document}

\newcommand{\normalstyle}{\baselineskip 4mm \parskip 2mm \normalsize}
\newcommand{\tablestyle}{\baselineskip 2mm \parskip 1mm \small }
\newcommand{\eg}{{\it e.g.,}}
\newcommand{\ie}{{\it i.e.,}}
\pagestyle{myheadings}
\thispagestyle{empty}

%
%

\vskip -.5cm
\pretolerance 10000 
\listparindent 0cm
\labelsep 0cm
%
%

\vskip -30pt
%\maketitle
\vskip -30pt
\normalstyle
\pagestyle{empty}
\huge

\hphantom{aaa}
\vskip 25pt

\centerline{\Huge Representations of Celestial Coordinates in FITS}
\vskip 50pt
\centerline{\Huge Eric W. Greisen}
\vskip 20pt
\centerline{\huge National Radio Astronomy Observatory}
\vskip 30pt
\centerline{\Huge Mark Calabretta}
\vskip 20pt
\centerline{\huge Australia Telescope National Facility}
\vfill\eject

\hphantom{aaa}
\vskip -25pt

\centerline{\Huge Initial (1979) FITS Coordinate Specification}
\vskip 20pt
\begin{center}
\begin{tabular}{l}
\multicolumn{1}{l}{\Huge Keywords} \\
\noalign{\vskip 6pt}
\hphantom{aa} {\tt CTYPE}{\it n}  Coordinate type (8 characters)\\
\hphantom{aa} {\tt CRPIX}{\it n}  Reference pixel location \\
\hphantom{aa} {\tt CRVAL}{\it n}  Coordinate value at reference pixel \\
\hphantom{aa} {\tt CDELT}{\it n}  Coordinate increment at reference pixel \\
\hphantom{aa} {\tt CROTA}{\it n}  Coordinate ``rotation'' \\
\noalign{\vskip 13pt}
\multicolumn{1}{l}{\Huge Problems with this} \\
\noalign{\vskip 6pt}
\hphantom{aa} $\bullet$  Types inadequate for celestial coordinates,
                             velocities\\ 
\hphantom{aa} $\bullet$  Physical meaning of reference pixel undefined \\
\hphantom{aa} $\bullet$  Rotation undefined, inadequate for general
                             use \\
\hphantom{aa} $\bullet$  No provision for skew in images \\
\noalign{\vskip 13pt}
\multicolumn{1}{l}{\Huge AIPS 1983 enhancements} \\
\noalign{\vskip 6pt}
\hphantom{aa} $\bullet$  Types defined for some celestial coordinates,
                            velocities \\
\hphantom{aa} $\bullet$  Reference pixel defined as tangent point\\
\hphantom{aa} $\bullet$  Rotation limited to celestial coordinates \\
\hphantom{aa} $\bullet$  Still no skew or offset rotations  \\
\hphantom{aa} $\bullet$  Limited view of projective geometries
\end{tabular}
\end{center}
\vfill\eject

\hphantom{aaa}
\vskip -25pt

\centerline{\Huge Proposed FITS Coordinate Specification}
\vskip 20pt
\centerline{\Huge New Keywords for All Coordinate Types}
\vskip 20pt
\begin{center}
\begin{tabular}{l}
\multicolumn{1}{l}{{\tt CUNIT}{\it n} coordinate units string} \\
\hphantom{aa} $\bullet$  the units used in {\tt CRVAL}{\it n} and
                             {\tt CDELT}{\it n} \\
\hphantom{aa} $\bullet$  simple SI units (and degrees) preferred \\
\hphantom{aa} $\bullet$  allowed units and format of string to be
                             discussed \\ 
\noalign{\vskip 9pt}
\multicolumn{1}{l}{{\tt PC{\it nnnmmm}} {\Huge Matrix}} \\
\hphantom{aa} $\bullet$  Replaces {\tt CROTA}{\it n} keywords \\
\hphantom{aa} $\bullet$  Allows skew, offset and general rotations \\
\hphantom{aa} $\bullet$  Allows dissimilar coordinates to be rotated
                            together \\
\hphantom{aa} $\bullet$  Relative coordinates given by linear
                            transform :
\end{tabular}
\end{center}
{\LARGE
\begin{displaymath}
\left ( \begin{array}{c} x \\ y \\ z \\ \vdots \end{array} \right ) =
   \left ( \begin{array}{cccc}
   \hbox{{\tt CDELT1}} & \hbox{{\tt 0}} & \hbox{{\tt 0}} & \ldots \\ 
   \hbox{{\tt 0}} & \hbox{{\tt CDELT2}} & \hbox{{\tt 0}} & \ldots \\ 
   \hbox{{\tt 0}} & \hbox{{\tt 0}} & \hbox{{\tt CDELT3}} & \ldots \\ 
   \vdots          & \vdots          & \vdots          & \ddots
   \end{array} \right )
   \left ( \begin{array}{cccc}
   \hbox{{\tt PC001001}} & \hbox{{\tt PC001002}} &
         \hbox{{\tt PC001003}} & \ldots \\ 
   \hbox{{\tt PC002001}} & \hbox{{\tt PC002002}} &
         \hbox{{\tt PC002003}} & \ldots \\ 
   \hbox{{\tt PC003001}} & \hbox{{\tt PC003002}}
         & \hbox{{\tt PC003003}} & \ldots \\ 
   \vdots          & \vdots          & \vdots          & \ddots
   \end{array} \right )
   \left ( \begin{array}{c}
   i - i_0 \\ j - j_0 \\ k - k_0 \\ \vdots \end{array} \right )
\end{displaymath}
}
\vfill\eject

\hphantom{aaa}
\vskip -25pt

\centerline{\Huge Proposed FITS Coordinate Specification}
\vskip 20pt
\centerline{\Huge New Keywords for All Coordinate Types}
\vskip 20pt
\begin{center}
\begin{tabular}{lll}
%\noalign{\vskip -7pt}
\noalign{\vskip 7pt}
\multicolumn{3}{l}{\Huge Secondary description(s) of coordinate axis
                  $n$}\\
\noalign{\vskip 7pt}
\hphantom{aa} $\bullet$ {\tt C{\it m\/}VAL{\it n}} & & coordinate
                  value at reference pixel \\ 
\hphantom{aa} $\bullet$ {\tt C{\it m\/}PIX{\it n}} & & reference pixel
                  array location \\
\hphantom{aa} $\bullet$ {\tt C{\it m\/}ELT{\it n}} & & coordinate
                  increment at reference pixel \\ 
\hphantom{aa} $\bullet$ {\tt C{\it m\/}YPE{\it n}} & & axis type (8
                  characters) \\
\hphantom{aa} $\bullet$ {\tt C{\it m\/}NIT{\it n}} & & units of {\tt
                  C{\it m\/}VAL{\it n}} and {\tt C{\it m\/}ELT{\it n}}
                  \\
                  &\ & (character valued as {\tt CUNIT}{\it m})\\
\noalign{\vskip 7pt}
\multicolumn{3}{l}{\hphantom{aa $\bullet$ aaaa} 
                   $m = 2, 3, 4, 5, 6, 7, 8, \hbox{\ or\ } 9$} \\
\noalign{\vskip 17pt}
\multicolumn{3}{l}{\Huge Astrometry-related keywords added} \\
\noalign{\vskip 7pt}
\multicolumn{3}{l}{\hphantom{aa} $\bullet$ {\tt EQUINOX} replaces {\tt
                  EPOCH} for the epoch of the mean} \\
\multicolumn{3}{l}{\hphantom{aa $\bullet$ aa}  equator and equinox in
                  years} \\
\multicolumn{3}{l}{\hphantom{aa} $\bullet$ {\tt MJD-OBS} modified
                  Julian date of observation in days} \\
\multicolumn{3}{l}{\hphantom{aa} $\bullet$ {\tt RADECSYS} frame of
                  reference of equatorial coordinates} \\
\multicolumn{3}{l}{\hphantom{aa $\bullet$ aa} as {\tt FK4}, {\tt
                  FK4-NO-E}, {\tt FK5}, {\tt GAPPT}} \\
\end{tabular}
\end{center}
\vfill\eject

\hphantom{aaa}
\vskip -25pt

\centerline{\Huge Proposed FITS Coordinate Specification}
\vskip 20pt
\centerline{\Huge Keywords for Celestial Coordinates}
\vskip 20pt
\begin{center}
\begin{tabular}{l}
\multicolumn{1}{l}{{\tt CTYPE{\it n}} {\Huge format defined}} \\
\hphantom{aa} $\bullet$  First 4 characters give type of ``standard
                           coordinate system'' as \\
\hphantom{aa $\bullet$ aa} equatorial --- {\tt RA--} and {\tt DEC-} \\
\hphantom{aa $\bullet$ aa} galactic --- {\tt GLON} and {\tt GLAT} \\
\hphantom{aa $\bullet$ aa} ecliptic --- {\tt ELON} and {\tt ELAT} \\
\hphantom{aa} $\bullet$  Second 4 characters give type of projection
                            as {\tt -{\it ccc}} \\
\hphantom{aa $\bullet$ aa} where {\it ccc} defined by convention, such
                           as \\
\hphantom{aa $\bullet$ aa} {\tt SIN}, {\tt TAN}, {\tt ARC}, {\tt AIT} \\
\noalign{\vskip 12pt}
\multicolumn{1}{l}{{\tt CRPIX{\it n}} {\Huge meaning defined by
                           projection }} \\
\hphantom{aa} $\bullet$  Each projection has ``native coordinate
                           system '' \\
\hphantom{aa} $\bullet$  Reference pixel is native north pole
                           $(0,90^{\circ})$ for \\
\hphantom{aa $\bullet$ aa} azimuthal and conical projections \\
\hphantom{aa} $\bullet$  Reference pixel is native origin $(0,0)$ for \\
\hphantom{aa $\bullet$ aa} cylindrical and conventional projections \\
\end{tabular}
\end{center}
\vfill\eject

\hphantom{aaa}
\vskip -25pt

\centerline{\Huge Proposed FITS Coordinate Specification}
\vskip 20pt
\centerline{\Huge Celestial Coordinates Continued}
\vskip 20pt
\begin{center}
\begin{tabular}{l}
\multicolumn{1}{l}{{\tt CDELT{\it n}} {\Huge clarified}} \\
\hphantom{aa} $\bullet$  Increment in physical units per pixel of
                         physical axis $n$ \\
\hphantom{aa} $\bullet$  Applied after pixel rotation and skew \\
\hphantom{aa} $\bullet$  This linear physical ``coordinate'' then
                         converted by \\
\hphantom{aa $\bullet$}  non-linear formul\ae\ to true physical
                         coordinates \\
\noalign{\vskip 13pt}
\multicolumn{1}{l}{{\tt CRVAL{\it n}} {\Huge clarified}} \\
\hphantom{aa} $\bullet$  Value at the reference pixel in the standard
                         coordinates \\
\hphantom{aa $\bullet$} specified in {\tt CTYPE}{\it n} \\ 
\noalign{\vskip 13pt}
\multicolumn{1}{l}{{\tt LONGPOLE} \Huge keyword added for generality}
                        \\
\hphantom{aa} $\bullet$ Native longitude of north pole of standard
                        system \\
\hphantom{aa} $\bullet$ Default value is $180^{\circ}$ \\
\noalign{\vskip 10pt}
\multicolumn{1}{l}{{\tt PROJP{\it j}} {\Huge keywords added to define
                        some projections }} \\
\end{tabular}
\end{center}
\vfill\eject

\hphantom{aaa}
\vskip -25pt

\centerline{\Huge Proposed FITS Coordinate Specification}
\vskip 20pt
\centerline{\Huge Conventions and Matters of ``Good Form''}
\vskip 20pt
\begin{center}
\begin{tabular}{rl}
1. & The center of each pixel is its location. \\
\noalign{\vskip 5pt}
2. & Default viewing convention: \\
   &\hphantom{a} $\bullet$ First pixel at lower left corner, \\
   &\hphantom{a} $\bullet$ First axis displayed along horizontal, \\
   &\hphantom{a} $\bullet$ Second axis displayed along vertical. \\
\noalign{\vskip 5pt}
3. & Diagonal elements of {\tt PC{\it iiijjj}} should predominate.\\ 
   &\hphantom{a} $\bullet$ Do not hide transpositions in the {\tt PC}
                             matrix. \\
\noalign{\vskip 5pt}
4. & Forbid rotation into axes which have only integral values.\\
\noalign{\vskip 5pt}
5. & {\tt NCP} projection (of WSRT) changed to offset {\tt SIN}
        projection.\\ 
\noalign{\vskip 5pt}
6. & When possible, {\tt CROTA}{\it n} should be written along with
        {\tt PC{\it iiijjj}}. \\
\noalign{\vskip 5pt}
7. & For longitude axis $i$ and latitude axis $j$, the conversion is\\
\end{tabular}
\end{center}
\vskip -5pt
\begin{displaymath}
\left ( \begin{array}{cc}
   \hbox{{\tt PC{\it iiiiii}}} & \hbox{{\tt PC{\it iiijjj}}} \\
   \hbox{{\tt PC{\it jjjiii}}} & \hbox{{\tt PC{\it jjjjjj}}}
   \end{array} \right ) =
\left ( \begin{array}{cc}
    \cos(\hbox{{\tt CROTA{\it j}}}) &
  - \sin(\hbox{{\tt CROTA{\it j}}}) \\ 
    \sin(\hbox{{\tt CROTA{\it j}}}) &
    \cos(\hbox{{\tt CROTA{\it j}}})
   \end{array} \right ) \, .
\end{displaymath}
\vfill\eject

\end{document}
