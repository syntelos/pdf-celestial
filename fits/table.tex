\documentstyle{article}
\setlength{\textwidth}{135mm}
% process this file with standard LaTeX. D.Wells, NRAO-CV, 27Jun90.

\begin{document}

\begin{center}


This text was prepared for an appendix to a volume\\
produced in the Comp. Sci. Dept.  of the Univ. of Virginia\\
to document a workshop on scientific databases held in April 1990.\\[30mm]

{\large\bf FITS --- A Self-Describing Table Interchange Format}\\[10mm]

Donald C.~Wells\footnote{{\tt dwells@nrao.edu}; 804-296-0277; Donald
C.~Wells, National Radio Astronomy Observatory, Edgemont Road,
Charlottesville, VA 22903-2475.}\\[5mm]

National Radio Astronomy Observatory\footnote{NRAO is operated by Associated
Universities, Inc., under agreement with the National Science
Foundation.}\\[5mm]

27 June 1990\\[20mm]

\end{center}

Basic FITS, the ``Flexible Image Transport System'', is a data format
which was designed by astronomers in 1979\footnote{Wells, D.C. and
Greisen, E.W., 1981, Astron. Astrophys. Suppl. Ser. {\bf 44}, 363-370,
``FITS: A Flexible Image Transport System''.} to support interchange
of {\em n\/}-dimensional integer and floating point matricies using a
self-describing notation.  FITS is the de facto interchange format
used by astronomers everywhere since 1980.  The rules of the format
are controlled by the FITS Working Group of Commission~5 (Astronomical
Data) of the International Astronomical Union, and there are North
American and European FITS standards Committees as well. FITS is also
the official interchange and archive format for NASA astrophysics
missions, and NASA operates a FITS Support Office, including a
hot-line service.\footnote{Barry Schlesinger, 301-794-4246, {\tt
bschlesinger@ncf.span.nasa.gov}} There is an anonymous-guest archive
for FITS matters\footnote{{\tt fits.cx.nrao.edu}, 192.33.115.8, in
directory {\tt /FITS}; this text is {\tt /FITS/doc/fitsdbmsapp.tex}}
and an E-mail exploder.\footnote{send requests to be added to the
mailing list to {\tt fitsbits-request@fits.cx.nrao.edu}}

The architecture of Basic FITS is extensible; the meta-rules for
extensions are also a part of the standard.\footnote{Grosb{\o}l, P.,
Harten, R.H., Greisen, E.W. and Wells, D.C., 1988, Astron. Astrophys.
Suppl. Ser. {\bf 73}, 359-364, ``Generalized Extensions and Blocking
Factors for FITS''.} In particular, extensions to transmit tables have
been designed, and the ASCII tables extension is also a part of the
FITS standard.\footnote{Harten, R.H., Grosb{\o}l, P., Greisen, E.W.
and Wells, D.C., 1988, Astron. Astrophys.  Suppl. Ser.  {\bf 73},
365-372, ``The FITS Tables Extension''.} Numerous CDROMs containing
databases in the FITS ASCII tables format have been published by NASA
projects during the past two years.  A binary tables extension to FITS
has been proposed;\footnote{Cotton, W.D., 1990, ``FITS Binary Tables'',
draft available from D.~Wells.} prototype implementations have
demonstrated interoperability and this extension is currently being
considered by the FITS committees for adoption.

The FITS ASCII tables extension is capable of conveying a set of
tables as a self-documenting machine-independent and OS-independent
bytestream. The logical record size is 2880~bytes;\footnote{This
(peculiar) size is rich in prime factors; it is commensurate with the
word and byte sizes of all computers that have ever been sold in the
commercial market. In 1979, when FITS was designed, machines with
6-bit bytes and 24-, 36-, and 60-bit word sizes and ones-complement
arithmetic were still commonly used by astronomers.  Indeed, the first
FITS file was written by an IBM~360 (32-bit, twos-complement, EBCDIC
codes, PL/I program) and was read by a CDC~6400 (60-bit,
ones-complement, 6-bit ``Display'' codes, Fortran program); that
interchange worked on the first try and the file is still readable
today by all FITS readers, long after both original environments have
become irrelevant to astronomical computing. Obviously a new format
design today would use record lengths of $2^{n}$ but most astronomers
believe that the principle of protecting the older bits is still
important.} record blocking by integer factors from one to some limit
(typically ten) is allowed on media for which it is a relevant
concept.\footnotemark[7] The data structures are preceded by
``headers'', which are 80-character lines in keyword-equals-value
format. There are 36 header lines per logical record, and records are
padded with blanks. FITS headers and tables extensions do not contain
carriage returns or line feeds or other non-printing ASCII codes.

The FITS ASCII tables extensions are appended to the Basic FITS binary
matrix.  The matrix dimensions are allowed to be zero, but the minimum
Basic FITS header must still be present. There are two reasons for
this convention: (1)~FITS is a {\em family} of formats which have
internal consistency, and this simplifies documentation, shortens
learning time and made standards negotiations easier, and (2)~in many
scientific applications auxiliary tabular data structures need to be
associated with the main binary matrix data structures.

In this appendix we will display a typical FITS table, a single
table in a FITS file. The table has 2268 rows and 22 columns
encoded in 80 ASCII characters. The data were produced by automatic
software which searched images of the Northern sky produced from scans
made by the 300-foot telescope at Green Bank, WV (the 300-foot
collapsed in November 1988, about a year after these data were
recorded), and were given to D.~Wells by James J.~Condon of NRAO for
use in this appendix.  In the verbatim listings shown below two extra
lines have been prefixed to the listing of each logical record to show
the column alignments in the file, and the record number and line
number are shown for each line (these are not a part of the FITS
bytestream, of course). First, we show the minimum Basic FITS header:

\pagebreak
{\footnotesize \begin{verbatim}
                 1         2         3         4         5         6         7         8
 r/l    12345678901234567890123456789012345678901234567890123456789012345678901234567890

01/01:  SIMPLE  =                    T / Standard FITS format (AA Suppl. 73, 365)       
01/02:  BITPIX  =                    8 / 8-bit characters                               
01/03:  NAXIS   =                    0 / No image data array present                    
01/04:  EXTEND  =                    T / There may be standard extensions               
01/05:  BLOCKED =                    T / Tape may be blocked (2880 byte records)        
01/06:  TELESCOP= 'NRAO91M '           / 91m = 300-ft telescope (r.i.p.)                
01/07:  INSTRUME= '7BEAM6CM'           / 7-beam receiver                                
01/08:  OBJECT  = '87GB CAT'           / The 87GB 4.85 GHz source catalog               
01/09:  EPOCH   =               1950.0 / Equinox (yr) of RA, dec values in table        
01/10:  DATE-OBS= '01/10/87'           / Observation start date (dd/mm/yy)              
01/11:  OBSERVER= 'CBS     '           / Condon, Broderick, and Seielstad               
01/12:  ORIGIN  = 'NRAOCV  '           / Written at NRAO, Charlottesville               
01/13:  DATE    = '11/06/90'           / Date file written (dd/mm/yy)                   
01/14:  HISTORY  AIPS IMNAME = 'B1950.11H'                                              
01/15:                                                                                  
01/16:  COMMENT  This table contains all sources from the 87GB catalog                  
01/17:  COMMENT  with hours of right ascension = 11 (equinox B1950)                     
01/18:  COMMENT  derived from the Green Bank 4.85 GHz sky survey made in 1987           
01/19:  COMMENT  with the 91-m telescope  (J. J. Condon, J. J. Broderick, and           
01/20:  COMMENT  G. A. Seielstad 1989, A. J. 97, 1064),                                 
01/21:  COMMENT  in standard FITS table format (see Astr. Ap. Suppl. 73, 365).          
01/22:  COMMENT  Catalog reference:  P. C. Gregory and J. J. Condon,                    
01/23:  COMMENT  Ap. J. Suppl., submitted May 1990.                                     
01/24:  END                                                                             
01/25:                                                                                  
01/26:                                                                                  
01/35:                                                                                  
01/36:                                                                                  
\end{verbatim} }

The {\tt NAXIS} line declares that the binary matrix does not exist
(it's dimensionality, the number of axes, is zero); the type of the
matrix elements ({\tt BITPIX=8}) does not matter in this case. The
following keyword-value pairs of this header are optional (except the
{\tt END}). The next logical record begins the header of the ASCII
table:

{\footnotesize \begin{verbatim}
                 1         2         3         4         5         6         7         8
 r/l    12345678901234567890123456789012345678901234567890123456789012345678901234567890

02/01:  XTENSION= 'TABLE   '           / Table extension                                
02/02:  BITPIX  =                    8 / 8-bit characters                               
02/03:  NAXIS   =                    2 / Simple 2-D matrix                              
02/04:  NAXIS1  =                   80 / Number of characters per row                   
02/05:  NAXIS2  =                 2268 / Number of rows = number of sources             
02/06:  PCOUNT  =                    0 / No random parameters                           
02/07:  GCOUNT  =                    1 / Only one group                                 
02/08:  TFIELDS =                   22 / Number of fields per row                       
02/09:  EXTNAME = 'B1950.11H'          / Name (Epoch.hours of right ascension)          
02/10:  EXTVER  =                    1 / Version number                                 
02/11:  EXTLEVEL=                    1 / Hierarchical level                             
02/12:                                                                                  
02/13:  TTYPE1  = 'RAH     '           / right ascension (hours)                        
02/14:  TBCOL1  =                    1 / start in column 1                              
02/15:  TFORM1  = 'I2      '           / 2-digit integer                                
02/16:  TUNIT1  = 'HR      '           / units are hours                                
02/17:  TNULL1  = '99      '                                                            
02/18:                                                                                  
02/19:  TTYPE2  = 'RAM     '           / right ascension (minutes)                      
02/20:  TBCOL2  =                    3 / start in column 3                              
02/21:  TFORM2  = 'I2      '           / 2-digit integer                                
02/22:  TUNIT2  = 'MIN     '           / minutes of time                                
02/23:  TNULL2  = '99      '                                                            
02/24:                                                                                  
02/25:  TTYPE3  = 'RAS     '           / right ascension (seconds)                      
02/26:  TBCOL3  =                    5 / start in column 5                              
02/27:  TFORM3  = 'E4.1    '           / xx.x SP floating point                         
02/28:  TUNIT3  = 'S       '           / seconds of time                                
02/29:  TNULL3  = '99.9    '                                                            
02/30:                                                                                  
02/31:  TTYPE4  = 'URAS    '           / rms uncertainty in RAS (seconds)               
02/32:  TBCOL4  =                   10 / start in column 10                             
02/33:  TFORM4  = 'E3.1    '           / x.x SP floating point                          
02/34:  TUNIT4  = 'S       '           / seconds of time                                
02/35:                                                                                  
02/36:  TTYPE5  = 'DECDSIGN'           / declination sign                               
\end{verbatim} }

This is an extension of type ``{\tt TABLE}''; it is a 2-dimensional
matrix of 8-bit bytes, with 80 bytes per row and 2268 rows in the
matrix.  The keyword {\tt TFIELDS} on line~8 tells us that there are
22 columns in the table. Keyword {\tt EXTNAME} specifies a name for
this extension (multiple extension structures can be concatenated
within a single FITS file, and can be distinguished by their names).
Each of the table columns is documented by a set of five keywords.
{\tt TTYPE}{\em ii} specifies the column label for the {\em ii}-th
column.  {\tt TBCOL}{\em ii} specifies the ordinal in the matrix of
the first character of the data field of the column, and {\tt
TFORM}{\em ii} specifies the format (and the field width) in Fortran
style. {\tt TUNIT}{\em ii} specifies the physical units of the column
and {\tt TNULL}{\em ii} specifies the field value that signifies
nulls. Here is the last header record:

{\footnotesize \begin{verbatim}
                 1         2         3         4         5         6         7         8
 r/l    12345678901234567890123456789012345678901234567890123456789012345678901234567890

05/01:                                                                                  
05/02:  TTYPE20 = 'ZERO    '           / zero-level of fit (Jy)                         
05/03:  TBCOL20 =                   67 / start in column 67                             
05/04:  TFORM20 = 'E3.3    '           / (.)xxx SP floating point                       
05/05:  TUNIT20 = 'JY      '           / Jansky                                         
05/06:                                                                                  
05/07:  TTYPE21 = 'PIXX    '           / x-coordinate pixel number                      
05/08:  TBCOL21 =                   71 / start in column 71                             
05/09:  TFORM21 = 'I4      '           / 4-digit integer                                
05/10:                                                                                  
05/11:  TTYPE22 = 'PIXY    '           / y-coordinate pixel number                      
05/12:  TBCOL22 =                   76 / start in column 76                             
05/13:  TFORM22 = 'I4      '           / 4-digit integer                                
05/14:                                                                                  
05/15:  AUTHOR  = 'P. C. Gregory and J. J. Condon'                                      
05/16:  REFERENC= 'Ap. J. Suppl., submitted 1990 May'                                   
05/17:  DATE    = '11/06/90'           / file generation data (dd/mm/yy)                
05/18:                                                                                  
05/19:  END                                                                             
05/20:                                                                                  
05/35:                                                                                  
05/36:                                                                                 
\end{verbatim} }

The next FITS logical record begins the table itself, which extends
for $63 (= 2268 / 36)$ FITS records. Here are the first and last rows
of the table:

{\footnotesize \begin{verbatim}
                 1         2         3         4         5         6         7         8
 r/l    12345678901234567890123456789012345678901234567890123456789012345678901234567890

06/01:  11 0 1.5 1.0 +181916 19  63.2 226.9    67   10      1.13 0.96 -64  -3  513  362 
06/02:  11 0 3.2 1.5 +322917 23  65.7 194.2    35    7      1.38 0.68  55  -1  512  735 
06/03:  11 0 5.6 1.6 +27 6 9 26  65.6 207.3    30    7      1.58 0.75  44   0  511  253 
06/04:  11 0 9.7 2.0 +451913 25  61.8 166.3    27    6      1.28 0.69 -55  -2  510   93 
06/05:  11 011.0 1.1 +105913 20  59.4 240.1    67   11      1.15 0.67  -4  -6  509  601 
06/06:  11 011.7 0.9 + 515 9 18  55.8 248.3   135   20      1.22 0.66 -10   1  509   86 

68/04:  115859.9 2.1 + 917 1 44  68.3 267.0    40    9   W  1.94 1.32 -32  -4   92  450 
68/05:  1159 5.1 1.5 +472035 19  67.7 146.2    50    8      0.87 0.69 -80   1  222  283 
68/06:  1159 6.2 1.6 +231311 27  77.8 230.2    35    7      1.20 0.73 -71  -1  119  804 
68/07:  1159 6.9 0.9 +113140 19  70.1 263.6   146   21      0.92 0.74 -51  -2   92  651 
68/08:  115911.4 0.9 +144814 19  72.7 257.3   211   29      1.01 0.81 -59  -1   96  942 
68/09:  115914.0 0.9 +105335 19  69.6 264.7   153   22      1.12 0.68   5   0   89  594 
68/10:  115915.7 2.0 +502524 24  65.0 142.4    32    7      1.66 0.98 -65  -6  237  559 
68/11:  115916.7 1.0 +393541 16  73.9 160.6   261   32      1.15 0.85   8   1  179  484 
68/12:  115919.2 1.1 +3313 5 19  77.6 181.9    73   11      1.03 0.72  -8   1  150  806 
68/13:  115920.1 1.2 +445634 17  69.8 149.6    90   11      0.87 0.71   7   0  205  960 
68/14:  115920.7 1.1 +3651 2 18  75.7 168.4    86   12      1.24 0.96  88  -3  165  238 
68/15:  115929.5 1.4 +3130 0 23  78.3 189.5    43    8      0.99 0.67 -84   0  140  653 
68/16:  115930.2 1.2 +581838 14  57.9 135.5   268   27      1.09 0.81  -4   0  283  369 
68/17:  115933.9 1.9 +504917 22  64.7 141.8    36    6      0.94 0.79 -13  -2  235  595 
68/18:  115934.4 1.0 +13 228 20  71.4 261.1   107   16      1.15 0.84 -25   1   85  786 
68/19:  115935.0 2.3 +595826 22  56.3 134.4    35    6      1.77 0.99  89  -5  293  518 
68/20:  115936.1 1.2 +45 122 17  69.7 149.4    78   10      0.96 0.81 -78  -5  202   75 
68/21:  115937.6 1.6 +342824 25  77.1 176.7    36    7      1.34 0.70 -43   0  149  918 
68/22:  115939.8 1.0 +214047 19  77.2 236.8   116   16      1.04 0.72  25  -1  102  667 
68/23:  115941.5 2.5 +552123 26  60.6 137.6    27    6      1.25 0.83 -48  -5  262  104 
68/24:  115941.8 2.2 +622451 19  54.0 132.9    43    6      1.12 0.78  48  -2  308  736 
68/25:  115942.0 1.4 +182235 25  75.3 248.3    44    8      1.33 0.94 -85  -3   93  372 
68/26:  115942.6 1.8 +721656 11  44.6 128.3   140   12      1.07 0.79 -20  -1  378  722 
68/27:  115946.2 1.4 + 85619 26  68.1 267.9    48   10      1.19 0.61 -22   0   74  419 
68/28:  115946.4 1.2 + 83314 24  67.7 268.4    61   11      0.95 0.61 -41   1   74  384 
68/29:  115951.3 2.4 +272523 41  78.9 210.0    32    8   W  2.04 1.31  37  -8  117  288 
68/30:  115951.6 2.3 +534640 25  62.1 138.9    29    6      1.05 0.76 -41  -1  249  858 
68/31:  115952.1 1.4 +141830 26  72.5 258.8    43    9      1.35 0.66 -70  -2   81  898 
68/32:  115952.1 1.3 +133037 25  71.8 260.4    48    9   W  1.05 0.56   4  -3   79  827 
68/33:  115954.1 2.2 +514810 26  63.8 140.7    28    6      1.40 0.87  18   0  236  682 
68/34:  115956.7 1.9 +482732 25  66.8 144.4    42    7  E   1.49 1.14  32  -5  216  384 
68/35:  115957.3 2.2 +142322 57  72.5 258.6    37    8   W  2.83 0.80  29  -5   79  905 
68/36:  115957.7 1.9 +585831 19  57.3 134.9    44    7      1.35 0.92 -40   1  282  429 
\end{verbatim} }

This FITS bytestream consists of 68 logical records: 1 Basic FITS
header record, 4 header records for the table header and 63 records
for the table itself. The fact that the last row of the table {\em
exactly} fills the $36^{th}$ line of the $63^{rd}$ record is
accidental; normally the last record is padded with blanks. Also, the
fact that the rows are 80 characters long, commensurate with 2880, is
peculiar to this table; other tables might have other row lengths.  If
the row length is not commensurate the rows of the FITS matrix are
written as a contiguous stream without regard to logical record
boundaries.  The total stream is $195840 (=68\times 2880)$ bytes long.

This file is number 11 of 24, covering the eleventh hour of Right
Ascension, the celestial longitude coordinate, and the total survey
contains about 50000 sources.  Other analogous radio surveys at
different frequencies can be compared and composite tables containing
source strengths or non-detections as a function of frequency can be
constructed.  Similar surveys in other frequency ranges (X-ray,
ultraviolet, optical, infrared) can also be compared with this source
list, and valuable astrophysical insight comes from such
``panchromatic'' astronomy.

\end{document}