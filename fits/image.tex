%%%%%%%%%%%%%%%%%%%%%%%%%%%%%%%%%%%%%%%%%%%%%%%%%%%%%%%%%%%%%%%%%%%%%%%%%%%%%
%
%.Title: FITS Image Extension Proposal
%
%.Authors: J.D.Ponz, R.W. Thompson, J.R.Munoz    
%.History: 20 Dec 1991      First draft
%.History:  7 Feb 1992      Revision to include comments from B. Schlesinger
%%%%%%%%%%%%%%%%%%%%%%%%%%%%%%%%%%%%%%%%%%%%%%%%%%%%%%%%%%%%%%%%%%%%%%%%%%%%%
%

\documentstyle[11pt]{article}

%%%%%%%%%%%%%%%%%%%%%%%%%%%%%%%%%%%%%%%%%%%%%%%%%%%%%%%%%%%%%%%%%%%%%%%%%%
%\parindent=0truecm              % paragraph indentation
%\topmargin=-36pt                % top margin
\textwidth=14.5truecm              % page width
\textheight=22truecm             % length of the page
%\textheight=8in                  % length of the page
\topsep=0.5ex                    % space at the beginning of a list  h
\itemsep=0.0ex                   % separation of items in a list
%%%%%%%%%%%%%%%%%%%%%%%%%%%%%%%%%%%%%%%%%%%%%%%%%%%%%%%%%%%%%%%%%%%%%%%%%%

\title{The {\sl FITS} IMAGE Extension. \\
       A Proposal}
\author{J.D. Ponz (ESA), R.W. Thompson (CSC), J.R. Mu\~{n}oz (ESA)\\
TN/8017-01/JDP/920207}
\date{7 February 1992}

\begin{document}

\maketitle

\begin{abstract}
This paper describes the {\sl FITS} IMAGE extension which
provides a simple means for storing two or more related but not necessarily
identical arrays in a single {\sl FITS} file. 
The proposed extension is structured in a manner identical to that of a 
standard {\sl FITS} header and data unit, thereby making it simple to 
interpret and if desired, easily converted to separate {\sl FITS} primary 
array files.
\end{abstract}


\section{Introduction}
The IMAGE extension conforms to the generalized {\sl FITS} extension format
described by Grosb\o l et al (ref. 3). It was first proposed by Mu\~{n}oz 
(ref. 6)
for use by the International Ultraviolet Explorer (IUE) project,
as a means for combining image data with related auxiliary information
which could not be easily stored using GROUP format or merged with the
image data to create a single primary array. 
The IMAGE extension is a simple replication of the primary data
array with  the following features:

\begin{enumerate}
 \item Allows the storage of an unlimited number of multidimensional arrays.
 (Note: support for multidimensional arrays has been suggested for, but is not
 currently included in, the proposed binary table extension (ref. 7).
 \item Arrays are stored in separate extensions, thereby allowing
 each array to have its own header and semantic contents.
 \item {\sl FITS} readers can easily skip over reading separate extensions.
 \item No additional keywords or conventions need to be adopted.
 \item Dividing each individual header and data unit in a {\sl FITS} IMAGE
 extension file results in simple primary array format files without
 any further processing (other than replacing the 
 {\tt XTENSION = }\verb*+'IMAGE   '+ keyword with {\tt SIMPLE = T}).
\end{enumerate}

 The IMAGE extension is being proposed for use in the IUE Archive
reprocessing project, to store data quality flags associated with 
the spectral images.

\newpage

\section{Identification}
The IMAGE extension may appear in a {\sl FITS} file if the primary header of 
the {\sl FITS} file has the keyword {\tt EXTEND} set to {\tt T} and the first
keyword of the associated extension header has 
{\tt XTENSION = }\verb*+'IMAGE   '+.

No other modifications are required although {\tt NAXIS = 0}
is not recommended since it would not make sense to extend a non-existing 
image with another image.

\section{Extension Header}

The extension header would begin in the first byte of the first record
following the primary data array or the previous extension file. The
format is identical to the primary header format and is described in the
{\sl FITS} Draft standard (ref. 7). 
The content of the header should be sufficient to
allow a {\sl FITS} reader to decide if it should read or skip over the
extension. 

\subsection{Required Keywords}
The card images in the header of the IMAGE extension use the following
keywords in the order indicated in Table 1.

{\small
\begin{table}[htpb]
\begin{center}
\begin{tabular}{ll} \\
\multicolumn{1}{c}{Principal}  & \multicolumn{1}{c}{Proposed IMAGE} \\
\multicolumn{1}{c}{HDU}        & \multicolumn{1}{c}{Extension}      \\ \hline
{\tt SIMPLE}      & {\tt XTENSION}$^{1}$ \\
{\tt BITPIX}      & {\tt BITPIX}         \\
{\tt NAXIS}       & {\tt NAXIS}          \\
{\tt NAXISn}      & {\tt NAXISn}$^{2}$   \\
{\tt EXTEND}$^{3}$& {\tt PCOUNT = 0}     \\
{\tt END}         & {\tt GCOUNT = 1}     \\
                  & {\tt END}            \\
\hline
\end{tabular}
\end{center}
$^1$  \verb*+XTENSION= 'IMAGE   '+ for the proposed extension.\\
$^2$  {\tt n = 1, ..., NAXIS}.\\
$^3$  Required only if extension present.

\caption[Mandatory {\em FITS} keywords]                                        
         {Mandatory {\em FITS} keywords for the Principal HDU and the
          proposed IMAGE extension.}
\end{table}              
}

\subsubsection{{\tt XTENSION} Keyword}
The value field shall contain the character string \verb*+'IMAGE   '+.

\subsubsection{{\tt BITPIX} Keyword}
The value field shall contain an integer. The absolute value is used in 
computing the size of the data structure. 
It shall specify the number of bits that represent a data value.
All {\tt BITPIX} values allowed in the Primary Header for the primary 
data matrix are also allowed for the IMAGE extension header.

\subsubsection{{\tt NAXIS} Keyword}
The value field shall contain a non-negative integer no greater than 999,
representing the number of axes in the data array.

\subsubsection{{\tt NAXISn} Keywords}
The value field of this indexed keyword shall contain a non-negative 
integer, representing the number of positions along axis {\tt n}
({\tt n= 1, 2, ..., NAXIS}) of the data array. 

\subsubsection{{\tt PCOUNT} Keyword}
The value field shall contain a non-negative integer that shall be used 
in any way appropriate to define the data structure, consistent with the 
equation for {\tt NBITS} in the {\sl FITS} Draft Standard (ref. 7).
A simple  IMAGE extension will have {\tt PCOUNT = 0}.

\subsubsection{{\tt GCOUNT} Keyword}
The value field shall contain a non-negative integer that shall be used 
in any way appropriate to define the data structure, consistent with the 
equation for {\tt NBITS} in the {\sl FITS} Draft Standard (ref. 7).
A simple IMAGE extension will have {\tt GCOUNT  = 1}.

\subsubsection{{\tt END} Keyword}
This keyword has no associated value. Columns 9-80 shall be filled with ASCII
blanks.

{\small
\begin{table}[htpb]
\begin{center}
\begin{tabular}{lllll} \\ 
\multicolumn{1}{c}{Conforming} & \multicolumn{1}{c}{Bibliographic} & 
\multicolumn{1}{c}{Commentary} & \multicolumn{1}{c}{Observation}   & 
\multicolumn{1}{c}{Array} \\ 
\multicolumn{1}{c}{Extension} & \multicolumn{1}{c}{Keywords} & 
\multicolumn{1}{c}{Keywords}  & \multicolumn{1}{c}{Keywords}   & 
\multicolumn{1}{c}{Keywords} \\ \hline
{\tt EXTNAME} &{\tt AUTHOR}  & {\tt COMMENT}  &{\tt DATE-OBS}&{\tt BSCALE} \\
{\tt EXTVER}  &{\tt REFERENC}& {\tt HISTORY}  &{\tt TELESCOP}&{\tt BZERO}  \\
{\tt EXTLEVEL}&              &\verb*+        +&{\tt INSTRUME}&{\tt BUNIT}  \\
              &              &                &{\tt OBSERVER}&{\tt BLANK}  \\
              &              &                &{\tt OBJECT}  &{\tt CTYPEn} \\
              &              &                &{\tt EQUINOX} &{\tt CRPIXn} \\
              &              &                &{\tt EPOCH}$^{1}$&{\tt CROTAn} \\
              &              &                &              &{\tt CRVALn} \\
              &              &                &              &{\tt CDELTn} \\
              &              &                &              &{\tt DATAMAX}\\
              &              &                &              &{\tt DATAMIN}\\
\hline
\end{tabular}
\end{center}
$^1$  See comment on {\tt EPOCH} in ref. 7.

\caption[Reserved IMAGE extension keywords]
         {Reserved keywords for the proposed IMAGE extension.
          The keywords are defined in the {\em FITS} Draft Standard (ref.7).}
\end{table}
}

\newpage

\subsection{Reserved Keywords}
Additional keywords, located in the extension header after 
the {\tt GCOUNT} keyword and before {\tt END},
may be used to describe history of the data, 
characteristics of the observations, characteristics of the data array
and other information.

These keywords are optional but, when used to describe an extension, they shall 
have the meaning described in the {\sl FITS} Draft Standard (ref. 7).
Table 2 summarizes the reserved keywords.
                                                                               
        



\section{Data Sequence}

The data format is identical to that of a primary data array and
is described in detail in the {\sl FITS} Draft Standard (ref. 7). 
This format will allow each IMAGE extension to contain a single data array
of 1-999 dimensions with a data structure and scale factors
independent of other arrays.

The IMAGE extension data shall begin in the first byte of the first
record following the last header record. 
The first value of each subsequent row of the array shall 
immediately follow the last value of the previous row. 
In this manner, the array structure is independent of the record structure. 
Arrays of more than one dimension shall consist of
a sequence such that the index of the first dimension varies most rapidly,
and the index of the last dimension varies least rapidly.

\newpage

\section{Example IMAGE Extension Header}

The following is an example of how this type of extension can be used.
In this example, the primary data array contains an IUE
linearized image file with the associated pixel quality flags stored
using the IMAGE extension. 

{\small
\begin{verbatim}
                           Main Header

         1         2         3         4         5         6         7
123456789012345678901234567890123456789012345678901234567890123456789012345...
------------------------------------------------------------------------------
SIMPLE  =                    T / Standard FITS format
BITPIX  =                   16 / 2-Bytes, 2-s complement integers
NAXIS   =                    2 / Number of axes
NAXIS1  =                  768 / Number of pixels per row
NAXIS2  =                  768 / Number of rows
EXTEND  =                    T / Extensions may be present
CTYPE1  = 'SAMPLE  '           / X axis
CTYPE2  = 'LINE    '           / Y axis 
BSCALE  =           3.1250E-02 / REAL = (FITS * BSCALE) + BZERO
BZERO   =                   0. / Bias
ORIGIN  = 'VILSPA  '           / Institution generating tape
TELESCOP= 'IUE     '           / IUE telescope
FILENAME= 'SWP12345.LIHI'      / Filename (camera)(image).LI(disp)
DATE    = '12/10/92'           / Date tape was written as DD/MM/YY
...
END

                        Extension Header

         1         2         3         4         5         6         7
123456789012345678901234567890123456789012345678901234567890123456789012345...
------------------------------------------------------------------------------
XTENSION= 'IMAGE   '           / IMAGE extension
BITPIX  =                   16 / 2-Bytes, 2-s complement integers
NAXIS   =                    2 / Number of axes
NAXIS1  =                  768 / Number of pixels per row
NAXIS2  =                  768 / Number of rows
PCOUNT  =                    0 / Number of parameters per group 
GCOUNT  =                    1 / Number of groups
CTYPE1  = 'SAMPLE  '           / X axis
CTYPE2  = 'LINE    '           / Y axis
FILENAME= 'SWP12345.LFHI'      / Filename (camera)(image).LF(disp)
EXTNAME = 'LFHI    '           / Data quality flags 
...
END
\end{verbatim}
}

\newpage

\section{References}

\begin{enumerate}
 \item Wells, D.C., Greisen, E.W., Harten, R.H. 1981,
      "{\sl FITS}: A Flexible Image Transport System",
      {\sl Astron. Astrophys. Suppl.}, {\bf 44}, 363-370.
 \item Greisen, E.W., Harten, R.H. 1981,
      "An Extension of {\sl FITS} for Small Arrays of Data",
      {\sl Astron. Astrophys. Suppl.}, {\bf 44}, 371-374.
 \item Grosb\o l, P., Harten, R.H., Greisen, E.W., Wells, D.C. 1988,
      "Generalized Extensions and Blocking Factors for {\sl FITS}",
      {\sl Astron. Astrophys. Suppl.}, {\bf 73}, 359-364.
 \item Harten, R.H., Grosb\o l, P., Greisen, E.W., Wells, D.C. 1988,
      "The {\sl FITS} Table Extension",
      {\sl Astron. Astrophys. Suppl.}, {\bf 73}, 365-372.
 \item Wells, E. W., Grosb\o l, P. 1990, 
       "Floating Point Agreement for {\sl FITS}",
       (available from the NOST {\sl FITS} Support Office).
 \item Mu\~{n}oz, J.R. 1989, 
       "IUE Data in {\sl FITS} Format",
       ESA IUE Newsletter, {\bf 32}, 12-45.
 \item NOST 1991,
       "Implementation of the Flexible Image Transport System",
       NOST 100-0.3b Draft Standard.
       NASA/OSSA Office of Standards and Technology, 
       Goddard Space Flight Center.
\end{enumerate}

\end{document}
